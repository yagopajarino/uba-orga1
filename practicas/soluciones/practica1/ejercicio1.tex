\section*{Ejercicio 1}

\subsection*{a)}

Solo escribo el cálculo para el primero :)

\begin{align*}
    33 = 16 * 2 + &1 \\
    16 = 8 * 2 + &0 \\
    8 = 4 * 2 + &0 \\
    4 = 2 * 2 + &0 \\
    2 = 1 * 2 + &0 \\
    1 = 0 * 2 + &1
\end{align*}

Luego,
\begin{itemize}
    \item $33_{10} = 100001_{2} = 1020_{3} = 113_{5}$
    \item $100_{10} = 1100100_{2} = 10201_{3} = 400_{5}$
    \item $1023_{10} = 1111111111_{2} = 1101220_{3} = 13043_{5}$
\end{itemize}

Calculadora de cambio de base \url{https://www.rapidtables.com/convert/number/base-converter.html}

\subsection*{b)}

Solo escribo el cálculo para el primero :)

\begin{align*}
    1111_2 &= \left(\sum_{i = 0}^{4}d_i * 2^{i-1}\right)_{10} \\
    1111_2 &= 1 * 2^3 + 1 * 2^2 + 1 * 2^1 + 1 * 2^0 \\
    1111_2 &= 8 + 4 + 2 + 1 \\
    1111_2 &= 15 \\
\end{align*}

Luego,
\begin{itemize}
    \item $1111_2 = 15_{10}$
    \item $1111_3 = 40_{10}$
    \item $1111_5 = 156_{10}$
    \item $CAFE_{16} = 51966_{10}$
\end{itemize}

\subsection*{c)}

Acá hay que pasar primero a decimal y después a la base pedida.

\begin{itemize}
    \item $ 17_8 = 15_{10} = 30_5 $
    \item $ BABA_{13} = 26010_{10} = 320230_6 $
\end{itemize}

\subsection*{d)}

La estrategia para pasar de binario a bases potencia de 2 es agrupar digitos de a 2,3,4 digitos para las bases 4,8,16

\begin{align*}
    1001011010100101_2 &= [10][01][01][10][10][10][01][01] \\
    1001011010100101_2 &= [2][1][1][2][2][2][1][1] \\
    1001011010100101_2 &= 21122211_4 \\
\end{align*}

Luego,
\begin{itemize}
    \item $ 1001011010100101_2 = 21122211_4 = 113245_8 $
    \item $ 1111101100101101000001100111_2 = 33230231001213_4 = 1754550147_8 $
\end{itemize}
